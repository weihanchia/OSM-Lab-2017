\documentclass{article}

%%% SOME USEFUL PACKAGES %%%
\usepackage[english]{babel} % hyphenation
\usepackage[margin=1.5cm]{geometry} % margins
\usepackage{graphicx} % support for graphics
\usepackage{amsmath} % support for math­e­mat­i­cal typesetting
\usepackage{amssymb} % math­e­mat­i­cal symbols
\usepackage{color} % support for colors
\usepackage{mathtools} % more math­e­mat­i­cal type­set­ting
\usepackage{amsthm} % for defining theorem-like environments
\usepackage{enumerate} % change appearance of numbered lists
\usepackage{framed} % textboxes
\usepackage[format=plain,labelfont=bf,up]{caption} % cus­tomise cap­tions for fig­ures and ta­bles
\usepackage[colorlinks=true,linkcolor=black,urlcolor=blue,linktoc=all, citecolor=black]{hyperref} % hyperlinks
\usepackage{setspace}
\usepackage{verbatim}

%%% CUSTOM COMMANDS %%%
\def\ci{\perp\!\!\!\perp} % statistical independence symbol
\newcommand{\ind}{1\hspace{-2.1mm}{1}} % indicator function
\newcommand{\rl}{\mathbb{R}} % real numbers
\newcommand{\ex}[1]{\mathbb{E} \left\{ #1 \right\}} % expectation operator
\newcommand{\pr}[1]{\mathbb{P} \left\{ #1 \right\}} % probability
\newcommand{\var}[1]{\mathbb{V}\text{ar} \left\{ #1 \right\}} % variance
\newcommand{\cov}[1]{\mathbb{C}{ov} \left\{ #1 \right\}} % covariance
\newcommand{\corr}[1]{\mathbb{C}{orr} \left\{ #1 \right\}} % correlation
\newcommand{\inprod}[1]{\langle #1 \rangle}

\begin{document}
	\title{OSM Lab 2017: Math Pset 5}
	\author{Wei Han Chia}
	\date{Due: 21 July 2017}
	\maketitle
	
	\section*{Problems from the Book}
	\subsection*{7.1}
	We will prove that if $S$ is a non-empty subset of $V$, then $conv(S)$ is convex.
	
	Consider $z, y \in conv(S)$. Then we know by definition that $z = \sum_{i=1}^{n} \lambda_{i,z} x_i $ and $y = \sum_{i=1}^{n} \lambda_{i,y} x_i$. Now consider:
	\begin{align*}
	\lambda z + (1 - \lambda) y &= \lambda \sum_{i=1}^{n} \lambda_{i, z} + (1-\lambda) \sum_{i=1}^{n} \lambda_i x_{i,y} \\
	&= \sum_{i=1}^{n} (\lambda \lambda_{i,z} + (1-\lambda) \lambda_{i,y}) x_i 
	\end{align*}
	Now since $ \sum_{i=1}^{n} \lambda \lambda_{i,z} + (1-\lambda) \lambda_{i,y} = 1$, it follows that $\lambda z + (1 -\lambda)y$ is in $conv(S)$. As this holds for every $z,y$ and any $\lambda \in [0,1]$, it follows that $conv(S)$ is convex.
	
	\subsection*{7.2}
	(i) Consider $x, y \in P$, where $P$ is a hyperplane defined by $a, b$. Now:
	\begin{align*}
	\inprod{z = \lambda x + (1- \lambda) y, a} = \lambda \inprod{x, a} + (1- \lambda) \inprod{x, b} = b
	\end{align*}
	So $z \in P$, and therefore $P$ is convex.
	
	\noindent(ii)
	Consider $x, y, \in H$, where $H$ is again the half space defined by $a, b$. Now:
	\begin{align*}
	\inprod{z = \lambda x + (1- \lambda) y, a} = \lambda \inprod{x, a} + (1 -\lambda) \inprod{y,a} \leq b
	\end{align*} 
	So $z \in H$, and therefore $H$ is convex.
	
	\subsection*{7.4}
	We will use parts (i) - (iv) to prove hte following theorem. Let $C \subset \mathbb{R}^n$ be nonempty, closed and convex. A point $\mathbf{p} \in C$ is the projection of $\mathbf{x}$ onto $C$ iff
	\[ \inprod{\mathbf{x} - \mathbf{p}, \mathbf{p} -\mathbf{y}} \geq 0 \quad \forall \mathbf{y} \in C \]
	
	\noindent(i)
	\begin{align*}
	\Vert \mathbf{x} - \mathbf{y} \Vert^2 &= \inprod{\mathbf{x}-\mathbf{y}, \mathbf{x}-\mathbf{y}} \\
	&= \inprod{\mathbf{x} - \mathbf{p} + \mathbf{p} - \mathbf{y}, \mathbf{x} - \mathbf{p} + \mathbf{p} - \mathbf{y}} \\
	&= \inprod{\mathbf{x} - \mathbf{p}, \mathbf{x} - \mathbf{p}} + \inprod{\mathbf{p} - \mathbf{y}, \mathbf{p} - \mathbf{y}} + 2 \inprod{\mathbf{x} - \mathbf{p}, \mathbf{p} - \mathbf{y}} \quad \text{From the bilinearity and symmetry of inner products} \\
	&= \Vert \mathbf{x} - \mathbf{p} \Vert^2 + \Vert \mathbf{p} - \mathbf{y} \Vert^2 + 2 \inprod{\mathbf{x} - \mathbf{p}, \mathbf{p} - \mathbf{y}}
	\end{align*}
	
	\noindent(ii)
	From (i), we know that if $2 \inprod{\mathbf{x} - \mathbf{p}, \mathbf{p} - \mathbf{y}} \geq 0$, then
	\begin{align*}
	\Vert \mathbf{x} - \mathbf{y} \Vert^2 &\geq \Vert \mathbf{x} - \mathbf{p} \Vert^2 + \Vert \mathbf{p} - \mathbf{y} \Vert^2  \\
	&> \Vert \mathbf{x} - \mathbf{p} \Vert^2 \quad \text{Since } \Vert\mathbf{p} - \mathbf{y} \Vert^2 > 0 \text{ when }\mathbf{y} \neq \mathbf{p}
	\end{align*}
	This is ($\Rightarrow$)
	
	\noindent(iii)
	If $\mathbf{z} = \lambda \mathbf{y} + (1 - \lambda) \mathbf{p}$, $\lambda \in [0,1]$, then 
	\begin{align*}
	\Vert \mathbf{x} - \mathbf{z} \Vert^2 &= \Vert \mathbf{x} - \mathbf{p} \Vert^2 + \Vert \mathbf{p} - \mathbf{z} \Vert^2 + 2 \inprod{\mathbf{x} - \mathbf{p}, \mathbf{p} - \mathbf{z}} \\
	& \text{Substituting } \mathbf{z} = \lambda \mathbf{y} + (1- \lambda) \mathbf{p} \\
	&= \Vert \mathbf{x} - \mathbf{p} \Vert^2 + \lambda^2 \Vert \mathbf{y} - \mathbf{p} \Vert^2 + 2\lambda \inprod{\mathbf{x} - \mathbf{p}, \mathbf{p} - \mathbf{y}}
	\end{align*}
	
	\noindent(iv)
	Now let $\mathbf{p}$ be a projection of $\mathbf{x}$ onto the convex set $C$. Since $C$ is convex, for any $\mathbf{y} \in C$, we can define $\mathbf{z} = \lambda \mathbf{y} + (1 - \lambda) \mathbf{p} \in C$. 
	
	Now since $\mathbf{p}$ is a projection, we know that 
	\begin{align*}
	\Vert \mathbf{x} - \mathbf{p} \Vert &\leq \Vert \mathbf{x} - \mathbf{z} \Vert \\
	\Vert 0 &\leq \mathbf{x} - \mathbf{z} \Vert^2 - \Vert \mathbf{x} - \mathbf{p} \Vert^2 \\
	\text{From (iii)} \\
	&= \lambda^2 \Vert \mathbf{y} - \mathbf{p} \Vert^2 + 2 \lambda \inprod{\mathbf{x} - \mathbf{p}, \mathbf{p} - \mathbf{y}}  \\
	&= \lambda \Vert \mathbf{y} - \mathbf{p} \Vert^2 + 2 \inprod{\mathbf{x} - \mathbf{p}, \mathbf{p} - \mathbf{y}}
	\end{align*}
	Now since $\lambda \Vert \mathbf{y} - \mathbf{p} \Vert^2 > 0$, it follows that $\inprod{\mathbf{x} - \mathbf{p}, \mathbf{p} - \mathbf{y}} > 0$, and so we have $(\Leftarrow$).
	
	\subsection*{7.6}
	We will prove that if $f: \mathbb{R}^n \to \mathbb{R}$ is convex, then the set $S = \{ \mathbf{x} \mathbb{R}^n | f(\mathbf{x}) \leq c \} \subset \mathbb{R}^n$ is a convex set.
	
	\begin{proof}
		Consider any $\mathbf{x}_1, \mathbf{x}_2$ in $S$, and any $\lambda \in [0,1]$. Now:
		\begin{align*}
		f(\lambda \mathbf{x}_1 + (1-\lambda)\mathbf{x}_2) \leq \lambda f(\mathbf{x}_1) + (1 -\lambda) f(\mathbf{x}_2) \leq c 
		\end{align*}
		So clearly $\lambda \mathbf{x}_1 + (1-\lambda)\mathbf{x}_2$ is in $S$, and so $S$ is convex.
	\end{proof}
	
	\subsection*{7.7}
	We will prove that for any convex set $C$, and convex functions $f_1,...,f_k$ taking $C$ to $\mathbb{R}$, and for any $\lambda_1, ..., \lambda_k \geq 0$, the function 
	\[ f(\mathbf{x}) = \sum_{i=1}^{k} \lambda_i f_i(\mathbf{x}) \]
	is convex.
	
	\begin{proof}
		Consider any $\mathbf{x}_1, \mathbf{x}_2$ in $C$, $\lambda \in [0,1]$.
		\begin{align*}
		\lambda f(\mathbf{x}_1) + (1 -\lambda)f(\mathbf{x}_2) &= \lambda \sum_{i=1}^{k} \lambda_i f_i(\mathbf{x}_1) + (1 -\lambda) \sum_{i=1}^{k} \lambda_i f_i(\mathbf{x}_2) \\
		&= \sum_{i=1}^{k} \lambda_i (\lambda f_i(\mathbf{x}_1) + (1 - \lambda) f_i(\mathbf{x}_2)) \\
		&\geq \sum_{i=1}^{k} \lambda_i (f_i(\lambda \mathbf{x}_1 + (1 -\lambda)\mathbf{x}_2)) \\
		&= f(\lambda \mathbf{x}_1 + (1 -\lambda)(\mathbf{x}_2))
		\end{align*}
		And so $f$ is a convex function.
	\end{proof}
	
	\subsection*{7.13}
	Lets assume $f(x) < M$ and $f$ is convex. Now consider $x, y \in \mathbb{R}^n$. We will prove this by contradiction. If $f$ were not constant, then there exists some $x, y$ such that $f(x) > f(y)$.
	
	Then,
	$f(x) \leq \lambda(f(\lambda x + (1-\lambda)y)) + (1 - \lambda(f(y)))$.  But since $f(x) > f(y)$ ,we have $f(x) - f(y) + \lambda f(y) \leq \lambda(f(\lambda x + (1 -\lambda)y))$. Now note that as $\lambda \to \infty$, this implies that we have $\lim_{\lambda \to \infty} f(\lambda x + ( 1- \lambda y)) \geq \lim_{\lambda \to \infty} \frac{f(x) - f(y)}{\lambda} + f(y) = \infty$.
	
	However, this implies that $f$ is unbounded, a contradiction.
	
	\subsection*{7.20}
	We will prove that if $f: \mathbb{R}^n \to \mathbb{R}$ is convex and $-f$ is also convex, then $f$ is affine. 
	
	Consider $x, y \in \mathbb{R}^n$. 
	\begin{align*}
	f(\lambda x + (1 -\lambda)y) \leq \lambda f(x) + (1 -\lambda) f(y) \quad \text{Since $f$ is convex} \\
	f(\lambda x + (1 -\lambda)y) \geq \lambda f(x) + (1 -\lambda) f(y) \quad \text{Since $-f$ is convex} \\
	\implies f(\lambda x + (1 -\lambda)y) = \lambda f(x) + (1 -\lambda) f(y)
	\end{align*}
	Now clearly $f$ is a linear transformation, and so our function $f$ is an affine function with $L = f$ and $c = 0$. 
	
	\subsection*{7.21}
	We will show that if $D \subset \mathbb{R}$ with $f:\mathbb{R}^n \to D$, and if $\phi: D \to \mathbb{R}$ is a strictly increasing function, then $\mathbf{x}^*$ is a local minimizer for $\phi \circ f(\mathbf{x})$ subject to constraints $G and H$ if and only if $\mathbf{x}^*$ is a local minimizer for the $f(\mathbf{x})$ subject to constraints $G and H$
	
	$\Leftarrow$
	
	Now if $\mathbf{x}^*$ is a local minimizer of $\phi \circ f$, since $\phi$ is strictly increasing, this implies that for all $\mathbf{x}$ fulfilling our constraints, $f(\mathbf{x}) \geq f(\mathbf{x}^*)$, and so $\mathbf{x}^*$ is a local minimizer of $f$ subject to the same constraints.
	
	
	$\Rightarrow$
	
	Now if $\mathbf{x}^*$ is a local minimizer of $f$, then it follows that for all $\mathbf{x}$ subject to our constraints, $f(\mathbf{x}) \geq f(\mathbf{x}^*)$. Now since $\phi$ is strictly increasing, it follows that $\phi \circ f(\mathbf{x}) \geq \phi \circ f(\mathbf{x}^*)$ for all $\mathbf{x}$ subject to our constraints, and so it follows that $\mathbf{x}^*$ is a local minimizer of $\phi \circ f$ subject to our constraints.
	\end{document}