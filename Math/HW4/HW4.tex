\documentclass{article}

%%% SOME USEFUL PACKAGES %%%
\usepackage[english]{babel} % hyphenation
\usepackage[margin=1.5cm]{geometry} % margins
\usepackage{graphicx} % support for graphics
\usepackage{amsmath} % support for math­e­mat­i­cal typesetting
\usepackage{amssymb} % math­e­mat­i­cal symbols
\usepackage{color} % support for colors
\usepackage{mathtools} % more math­e­mat­i­cal type­set­ting
\usepackage{amsthm} % for defining theorem-like environments
\usepackage{enumerate} % change appearance of numbered lists
\usepackage{framed} % textboxes
\usepackage[format=plain,labelfont=bf,up]{caption} % cus­tomise cap­tions for fig­ures and ta­bles
\usepackage[colorlinks=true,linkcolor=black,urlcolor=blue,linktoc=all, citecolor=black]{hyperref} % hyperlinks
\usepackage{setspace}
\usepackage{verbatim}

%%% CUSTOM COMMANDS %%%
\def\ci{\perp\!\!\!\perp} % statistical independence symbol
\newcommand{\ind}{1\hspace{-2.1mm}{1}} % indicator function
\newcommand{\rl}{\mathbb{R}} % real numbers
\newcommand{\ex}[1]{\mathbb{E} \left\{ #1 \right\}} % expectation operator
\newcommand{\pr}[1]{\mathbb{P} \left\{ #1 \right\}} % probability
\newcommand{\var}[1]{\mathbb{V}\text{ar} \left\{ #1 \right\}} % variance
\newcommand{\cov}[1]{\mathbb{C}{ov} \left\{ #1 \right\}} % covariance
\newcommand{\corr}[1]{\mathbb{C}{orr} \left\{ #1 \right\}} % correlation
\newcommand{\inprod}[1]{\langle #1 \rangle}

\begin{document}
	\title{OSM Lab 2017: Math Pset 4}
	\author{Wei Han Chia}
	\date{Due: 14 July 2017}
	\maketitle
	
	\section*{Problems from the Book}
	\subsection*{6.1}
	Let us define the following functions.
	\begin{align*}
	f(\mathbf{w}) = e^{-\mathbf{w}^T \mathbf{x}} \\
	G(\mathbf{w}) = \mathbf{w}^T \mathbf{x} - \mathbf{w}^T A \mathbf{w} + \mathbf{w}^T A \mathbf{y} \\
	H(\mathbf{w}) = \mathbf{y}^T \mathbf{w} - \mathbf{w}^T \mathbf{x}
	\end{align*}
	
	Now we can write our optimization problem in the usual form:
	\begin{align*}
	\min \quad&- f(\mathbf{w}) \\
	s.t. \quad&-G(\mathbf{w}) \leq a \\
	&H(\mathbf{w}) = b
	\end{align*}
	
	\subsection*{6.5}
	We know that profits can be defined as $0.07 x + 0.05 y$, where $x$ is the number of milk bottles, and $y$ is the number of plastic knobs. We also have the following constraints on plastic ($4x + 3y \leq 240$) and on labor ($2x + y \leq 100$). We can write these constraints into a matrix:
	\begin{align*}
	A = \begin{bmatrix} 4 & 3 \\ 2 & 1 \end{bmatrix}
	\end{align*}
	
	We can then write our problem into the standard form of an optimization problem:
	\begin{align*}
	\min \quad& - 0.07x - 0.05y \\
	s.t. \quad& A \begin{bmatrix} x \\ y \end{bmatrix} \leq \begin{bmatrix} 240 \\ 100 \end{bmatrix}
	\end{align*}
	
	\subsection*{5.5}
	Consider the function 
	\[ f(x,y) = 3x^2y + 4xy^2 + xy \]
	We can take the partial derivatives and construct the Hessian
	\begin{align*}
	D[f(x,y)] = \begin{bmatrix} 6xy + 4y^2 + y & 3x^2 + 8xy + x \end{bmatrix} \\
	D^2[f(x,y)] = \begin{bmatrix} 6y & 8y + 6x + 1 \\ 6x + 8y + 1 & 8x \end{bmatrix}
	\end{align*}
	
	Solving the equation for $D[f(x,y)] = 0$ gives the following pairs of points, $(0,0),(0, -\frac{1}{4}), (-\frac{1}{3}, 0), (-\frac{1}{9}, -\frac{1}{12})$. We can then evaluate the Hessian to decide if these points are local maxima, minima or saddle points.
	
	\begin{align*}
	D^2[f(0,0)] &= \begin{bmatrix} 0 & 1 \\ 1 & 0 \end{bmatrix} \\
	D^2[f(0, \frac{-1}{4})] &= \begin{bmatrix} - 3/2 & - 1 \\ -1 & 0 \end{bmatrix} \\
	D^2[f(\frac{-1}{3}, 0)] &= \begin{bmatrix} 0 & -1 \\ -1 & -8/3 \end{bmatrix} \\
	D^2[f(\frac{-1}{9}, \frac{-1}{12})] &= \begin{bmatrix}  - 1/2 & - 1/3 \\ - 1/3 & -2/3 \end{bmatrix}
	\end{align*}
	
	Looking at the Hessian, we see that (0, -1/4) and (-1/9, -1/12) are both local maxima, since their Hessian is negative definite. In addition, we note that (0,0) and (-1/3, 0) are saddle points.
	
	\subsection*{6.11}
	Consider $f(x) = ax^2 + bx + c$, $a > 0, b, c \in mathbb{R}$. Note that the unique minimizer of $f$ is given by $x = -\frac{b}{2a}$. We can obtain this by solving the first order conditions and noting that $f''(x) > 0$. 
	
	Now consider the Newton iteration starting at any $x_0$.
	\begin{align*}
	x_1 &= x_0 - \frac{f'(x_0)}{f''(x_0)} \\
	&= x_0 - \frac{2a x_0 + b}{2a} \\
	&= \frac{-b}{2a}
	\end{align*}
	\end{document}